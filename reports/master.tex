
%%%%%%%%%%%%%%%%%%%%%%%%%%%%%%%%%%%%%%%%%%%%%%%%%%%%%%%%%%%%%%%%%%%%%%%%%%%%
% This will help you in writing your report
% Remember that the character % is a comment in latex

% Divide the work you have done in each of the labs 
% in a new chapter, as in the example below, 
% using a coherent title 


% Follows some examples to include HDL, figures, ... :

%-----------------------------
% VHDL file, using the sintax:


	%\begin{listato}
	%\lstinputlisting{./exeMPLE/listato1.vhd}
	%\end{listato}

% the path to the file must be correct, obviously
% Should you have listings written in other languages the method is the 
% same one, but the language set up must be changed using a different 
% setting for the command \lstset{language=VHDL} in file homebook.tex 


%-----------------------------
% figures in postcript (ps) or encapsulated postcript (eps)
% format, using the syntax:

%	\begin{figure}[h]
%	\centering
%	\includegraphics[width=9cm]{./cap1/figure1.eps}
%	\caption{Put a caption if you want (didascalia...:)))}
%	\label{put-a-label-for-referring-to-this-picture}	
%	\end{figure}

% the path to the file must be correct, obviously
% you can refer to this picture in any point of your document
% by typing the instruction:

% 	\ref{put-a-label-for-referring-to-this-picture}

% that is using the same label you put in the fiure label
% when you will run the "latex command" an automatic reference to
% this figure with the correct enumeration will be inserted


%-----------------------
% comment in text format (if you are not skilled in latex and don't want to be)
% using the sintax:

	%\begin{verbatim}
	% blablabla 
	%\end{verbatim}

% The verbatimg includes text as it is, as you could write in a normal text file 

% (BETTER) If uou want to write enhancing all the latex possibilities you 
% should add to you text a few commands in some particular cases. 
% In the following you have and example of a few chapters roughtly commented
% and written all in this file: remember that you can saparate
% each chapter in different files (this is always what a latex pro does) 
% and include them using the instruction: \input{./directoryxx/fileyy.tex}


%%%%%%%%%%%%%%%%%%%%%%%%%%%%%%%%%%%%%%%%%%%%%%%%%%%%%%%%%%%%%%%%%%%%%%%%%%%%%%%
%%%%%%%%%%%%%%%%%%%%%%%%%%%%%%%%%%%%%%%%%%%%%%%%%%%%%%%%%%%%%%%%%%%%%%%%%%%%%%%%%
%%%%%%%%%%%%%%%%%%%%%%%%%%%%%%
% Beginning of Document content

% Lab 1
\chapter{Lab 1: design and implementation of a digital filter}


%%%%%%%%%%%%%%%%%%%%%%%%%%%%%%%%%%%%%%%%%%%%%%%%%%%%%%%%%%%
% you can organize the report usign section -> \section{Reference model development}
% or subsection -> \subsection{Matlab model}

%%%%%%   First section
\section{Reference model development}

\subsection{Matlab model}

The frequency response of the filter is shown in Figure \ref{fig:lab1:fig1}.

\begin{figure}[h]
  \centering
  \includegraphics[width=3cm]{./logopoli_new}
  \caption{Caption for my first figure}
  \label{fig:lab1:fig1}
\end{figure}

The coefficients of the filter have one bit for the integer part and XYZ bits for the fractional part. They correspond to the integer values shown in Table \ref{tab:cap1:tab1}.

\begin{table}
  \centering
  \caption{Caption for my first table}
  \label{tab:cap1:tab1}  
  \begin{tabular}{c|c}
    $b_0$ & 10 \\
    $b_1$ & 115 \\
    \multicolumn{2}{c}{$\cdots$} \\
  \end{tabular}
\end{table}

\subsection{C model and THD}

The number of bits after each multiplication is XYZ and the THD is XYZ dB.

\subsection{Explanations, comparisons and comments}

As it can be observed ... bla, bla, bla, ...

%%%%%%   Second section
\section{VLSI implementation}

\subsection{Architecture}

The architecture of the filter is shown in Figure \ref{fig:lab1:fig2} and the timing diagram in Figure \ref{fig:lab1:fig3}.

\begin{figure}[h]
  \centering
  \includegraphics[width=3cm]{./logopoli_new}
  \caption{Caption for my figure}
  \label{fig:lab1:fig2}
\end{figure}

\begin{figure}[h]
  \centering
  \includegraphics[width=3cm]{./logopoli_new}
  \caption{Caption for my figure}
  \label{fig:lab1:fig3}
\end{figure}

\subsubsection{Simulation}

To process all the samples the simulation lasts XYZ ns. A snapshot from XYZ ns to KLM ns is shown in Figure \ref{fig:lab1:fig4} to highlight the behaviour of the filter when VIN moves from 0 to 1 and viceversa.

\begin{figure}[h]
  \centering
  \includegraphics[width=3cm]{./logopoli_new}
  \caption{Caption for my figure}
  \label{fig:lab1:fig4}
\end{figure}

\subsection{Logic synthesis}

Snapshots showing slack met and power consumption are shown in Figures 5, 6 and 7.

\begin{figure}[h]
  \centering
  \includegraphics[width=3cm]{./logopoli_new}
  \caption{Figures 5, 6 and 7}
%%  \label{fig:lab1:fig}
\end{figure}

Results of the synthesis are shown in Table \ref{tab:cap1:tab2}.

\begin{table}
  \centering
  \caption{Caption for my table: A is the area, P is the power consumption and T is the simulation time.}
  \label{tab:cap1:tab2}
  \begin{tabular}{c|c|c|c}
    $f_M$=\dots   & A=\dots & -       & T=\dots \\
    $f_M/2$=\dots & A=\dots & P=\dots & T=\dots \\
  \end{tabular}
\end{table}

\subsection{Place and route}

Snapshots showing no timing violation (timeDesign Summary table for both setup and hold modes) and power consumption are shown in Figures 8, 9 and 10.

\begin{figure}[h]
  \centering
  \includegraphics[width=3cm]{./logopoli_new}
  \caption{Figures 8, 9 and 10}
%%  \label{fig:lab1:fig}
\end{figure}

Results of the place and route are shown in Table \ref{tab:cap1:tab3}.

\begin{table}
  \centering  
  \caption{Caption for my table: A is the area, P is the power consumption and T is the simulation time.}
  \label{tab:cap1:tab3}
  \begin{tabular}{c|c|c|c}
    $f_M/2$=\dots & A=\dots & P=\dots & T=\dots \\
  \end{tabular}
\end{table}

\subsection{Explanations, comparisons and comments}

As it can be observed ... bla, bla, bla, ...

%%%%%% Third section
\section{Advanced architecture development}


%%%%%%%%%%%%%%%%%%%%%%%%%%%%%%
%%%%%%%%%%%%%%%%%%%%%%%%%%%%%%
% Lab 2
%%\chapter{Next chapter ...}
\chapter{Lab 2: digital arithmetic}


% include here only file for the third lesson and homeworks
% here the path to figures and VHDL should be ./cap2/

\section{FPU model}

\subsection{Simulation}

The FPU has been stimulated with the following numbers, corresponding to the hexadecimal configuration shown in Table \ref{tab:cap2:tab1}.

\begin{table}
  \centering
  \caption{Caption ...}
  \label{tab:cap2:tab1}  
  \begin{tabular}{cc|cc|cc}
    \hline
    \multicolumn{2}{c}{$a$} & \multicolumn{2}{c}{$b$} & \multicolumn{2}{c}{$r$} \\
    \hline 
    15 & 0x4b80 & 204 & 0x5a60 & 3060 & 0x69fa \\
    \multicolumn{6}{c}{\dots} \\
    \hline    
  \end{tabular}
\end{table}

Figure ... shows a snapshot of the simulation. As it can be observed stimulating the FPU with the numbers shown in Table \ref{tab:cap2:tab1} the results are the expected ones.

\subsection{Synthesis}

Table ... shows the results of the experiments required in the assignment.

\subsection{Explanations, comparisons and comments}

As it can be observed ... bla, bla, bla, ...

\subsection{R4-MBE multiplier}

The circuit implementing the R4-MBE is shown in Fig. ...
The Dadda-like tree is shown in Fig. ... with the required compressors. Moreover, the figure shows how sign extension has been simplified to avoid unnecessary compressors.

\subsubsection{Simulation}

\paragraph{Standalone multiplier}

Figure ... shows a snapshot of a simulation for the designed multiplier as a standalone block.

\paragraph{Whole FPU}

Figure ... shows a snapshot of a simulation for the designed multiplier included in the whole FPU.

\subsubsection{Synthesis}

The results of the synthesis are shown in ... (you can add a new table or put the results as part of the previous table).

\subsubsection{Explanations, comparisons and comments}

As it can be observed ... bla, bla, bla, ...



%%%%%%%%%%%%%%%%%%%%%%%%%%%%%%
%%%%%%%%%%%%%%%%%%%%%%%%%%%%%%
% Lab 3
\input{./lab3/lab3.tex}

%%%%%%%%%%%%%%%%%%%%%%%%%%%%%%
%%%%%%%%%%%%%%%%%%%%%%%%%%%%%%
% Lab 4
%\input{./lab4/lab4.tex}

