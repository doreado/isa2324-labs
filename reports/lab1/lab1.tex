\chapter{Lab 1: design and implementation of a digital filter}


%%%%%%%%%%%%%%%%%%%%%%%%%%%%%%%%%%%%%%%%%%%%%%%%%%%%%%%%%%%
% you can organize the report usign section -> \section{Reference model development}
% or subsection -> \subsection{Matlab model}

%%%%%%   First section
\section{Reference model development}

\subsection{Matlab model}

The frequency response of the filter is shown in Figure \ref{fig:lab1:fig1}.

\begin{figure}[h]
  \centering
  \includegraphics[width=3cm]{./logopoli_new}
  \caption{Caption for my first figure}
  \label{fig:lab1:fig1}
\end{figure}

The coefficients of the filter have one bit for the integer part and XYZ bits for the fractional part. They correspond to the integer values shown in Table \ref{tab:cap1:tab1}.

\begin{table}
  \centering
  \caption{Caption for my first table}
  \label{tab:cap1:tab1}  
  \begin{tabular}{c|c}
    $b_0$ & 10 \\
    $b_1$ & 115 \\
    \multicolumn{2}{c}{$\cdots$} \\
  \end{tabular}
\end{table}

\subsection{C model and THD}

The number of bits after each multiplication is XYZ and the THD is XYZ dB.

\subsection{Explanations, comparisons and comments}

As it can be observed ... bla, bla, bla, ...

%%%%%%   Second section
\section{VLSI implementation}

\subsection{Architecture}

The architecture of the filter is shown in Figure \ref{fig:lab1:fig2} and the timing diagram in Figure \ref{fig:lab1:fig3}.

\begin{figure}[h]
  \centering
  \includegraphics[width=3cm]{./logopoli_new}
  \caption{Caption for my figure}
  \label{fig:lab1:fig2}
\end{figure}

\begin{figure}[h]
  \centering
  \includegraphics[width=3cm]{./logopoli_new}
  \caption{Caption for my figure}
  \label{fig:lab1:fig3}
\end{figure}

\subsubsection{Simulation}

To process all the samples the simulation lasts XYZ ns. A snapshot from XYZ ns to KLM ns is shown in Figure \ref{fig:lab1:fig4} to highlight the behaviour of the filter when VIN moves from 0 to 1 and viceversa.

\begin{figure}[h]
  \centering
  \includegraphics[width=3cm]{./logopoli_new}
  \caption{Caption for my figure}
  \label{fig:lab1:fig4}
\end{figure}

\subsection{Logic synthesis}

Snapshots showing slack met and power consumption are shown in Figures 5, 6 and 7.

\begin{figure}[h]
  \centering
  \includegraphics[width=3cm]{./logopoli_new}
  \caption{Figures 5, 6 and 7}
%%  \label{fig:lab1:fig}
\end{figure}

Results of the synthesis are shown in Table \ref{tab:cap1:tab2}.

\begin{table}
  \centering
  \caption{Caption for my table: A is the area, P is the power consumption and T is the simulation time.}
  \label{tab:cap1:tab2}
  \begin{tabular}{c|c|c|c}
    $f_M$=\dots   & A=\dots & -       & T=\dots \\
    $f_M/2$=\dots & A=\dots & P=\dots & T=\dots \\
  \end{tabular}
\end{table}

\subsection{Place and route}

Snapshots showing no timing violation (timeDesign Summary table for both setup and hold modes) and power consumption are shown in Figures 8, 9 and 10.

\begin{figure}[h]
  \centering
  \includegraphics[width=3cm]{./logopoli_new}
  \caption{Figures 8, 9 and 10}
%%  \label{fig:lab1:fig}
\end{figure}

Results of the place and route are shown in Table \ref{tab:cap1:tab3}.

\begin{table}
  \centering  
  \caption{Caption for my table: A is the area, P is the power consumption and T is the simulation time.}
  \label{tab:cap1:tab3}
  \begin{tabular}{c|c|c|c}
    $f_M/2$=\dots & A=\dots & P=\dots & T=\dots \\
  \end{tabular}
\end{table}

\subsection{Explanations, comparisons and comments}

As it can be observed ... bla, bla, bla, ...

%%%%%% Third section
\section{Advanced architecture development}
